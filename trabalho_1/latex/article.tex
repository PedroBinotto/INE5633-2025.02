\documentclass[12pt]{article}

\usepackage[T1]{fontenc}
\usepackage[a4paper, left=3cm, right=2cm, top=3cm, bottom=2cm]{geometry}
\usepackage[dvipsnames]{xcolor}
\usepackage[toc]{glossaries}
\usepackage{authblk}
\usepackage{cite}
\usepackage{fontspec}
\usepackage{graphicx}
\usepackage{hyperref}
\usepackage{relsize}
\usepackage{setspace}
\usepackage{subcaption}
\usepackage{verbatimbox}
\usepackage[english, portuguese]{babel}
    \addto\captionsportuguese{\renewcommand*\bibname{Referências}}
    \addto\captionsportuguese{\renewcommand*\contentsname{Sumário}}

\newcommand\myshade{85}
\newcommand{\pprime}{\ensuremath{^{\prime}}}
\newcommand{\RNum}[1]{\uppercase\expandafter{\romannumeral #1\relax}}
\renewcommand\Authsep{, }
\renewcommand\Authand{, }
\renewcommand\Authands{, e }
\providecommand{\keywords}[1]{%
  \small
  \textbf{\textit{\iflanguage{english}{Keywords}{Palavras-chave} ---}} #1%
}

\hypersetup{
    pdftitle   = {Título de Artigo, Subtítulo de Artigo},
    pdfauthor  = {Pedro Santi Binotto},
    pdfsubject = {Resumo em Português},
    linkcolor  = black,
    citecolor  = black,
    urlcolor   = black,
    colorlinks = true,
    filecolor  = black,
    linktoc    = page
}%

% \newglossaryentry{gls} {
%   name={GLS},
%   description={Glossary entry}
% }

\title{Atividade Prática \RNum{1} \\ [0.2em]\smaller{}Trabalho sobre Métodos de busca}

\author[1]{Pedro Santi Binotto [20200634]\thanks{\texttt{pedro.binotto@grad.ufsc.br}}}
\author[2]{Cauã Pablo Padilha [22100895]\thanks{\texttt{padilha.caua@grad.ufsc.br}}}
\author[3]{Felipe Jun Hatsumura [19206699]\thanks{\texttt{fjhats@gmail.com}}}
\author[4]{Gabriel Lemos da Silva [18200628]\thanks{\texttt{glemoss.dev@gmail.com}}}
\date{\today}
\affil[1]{Departamento de Informática e Estatística, Universidade Federal de Santa Catarina}

\makeglossaries

\begin{document}
\begin{titlepage}
\selectlanguage{portuguese} 
\maketitle
\thispagestyle{empty}

\end{titlepage}

\tableofcontents

\printglossary[title=Glossário, toctitle=Glossário]

\section{Q1}

\paragraph{Proposta ---} enunciado na \textit{Questão 1}:

\begin{quote}
Quais os métodos ou funções principais e suas relações com o algoritmo A*?
\end{quote}

\subsubsection{Solução --- \textbf{Q1}}

O código inicia lendo os inputs da entrada, utilizando a função \texttt{capture\_input()} para determinar o tamanho do tabuleiro, o nível de dificuldade e a heurística desejada, no retorno dessa função é chamada a função \texttt{validate\_input()} que ordena o tabuleiro para encontrar o estado alvo além de verificar se há uma solução possível ou se a entrada é inválida. Após isso, o tempo de execução iniciado e são passados para função \texttt{a\_star()} -- função que contém o algoritmo em si -- os seguintes parâmetros:

\begin{itemize}
  \item \textbf{g} - Grafo que serve de estrutura de dados para o algoritmo, contendo apenas o estado inicial
  \item \textbf{n} - Tamanho do tabuleiro
  \item \textbf{s} - Estado inicial
  \item \textbf{t} - Estado alvo
  \item \textbf{l} - Heurística a ser utilizada
\end{itemize}

Na função \texttt{a\_star()} então são definidos os conjuntos que armazenam os estados visitados e os estados abertos, bem como o número de estados abertos e o número máximo de estados abertos (fronteira máxima). Nessa mesma função é executado em um laço de repetição o algoritmo até que o estado atual (\texttt{current}) seja o estado alvo (\texttt{t}), nesse momento é retornado as seguintes informações:

\begin{itemize}
  \item \textbf{open} - Conjunto de estados abertos
  \item \textbf{open\_upper\_bound} - Número máximo de estados abertos simultâneamente
  \item \textbf{path} - caminho descoberto pelo algoritmo para chegar ao estado destino
  \item \textbf{visited} - Conjunto de estados visitados
\end{itemize}

Por fim, o tempo de execução é parado e então são printados na tela os resultados e o tempo de execução do algoritmo.

\section{Q2}

\paragraph{Proposta ---} enunciado na \textit{Questão 2}:

\begin{quote}
Como foi gerenciada a fronteira, ou seja, quais verificações foram feitas antes de adicionar um estado na fronteira? Explicar e mostrar os respectivos trechos de código:
\end{quote}

\subsubsection{Solução --- \textbf{Q2}}

\paragraph{}
lakdalk

\section{Q3}

\paragraph{Proposta ---} enunciado na \textit{Questão 3}:

\begin{quote}
Descrição das heurísticas e comparação da faixa de valores e da precisão delas (no mínimo: dois casos difíceis, dois médios e um fácil); breve descrição sobre suas implementações.
\end{quote}

\subsubsection{Solução --- \textbf{Q3}}

Para atender os requisitos do trabalho foram feitas três heurísticas e o algoritmo sem heurísticas (custo uniforme – L0). As heurísticas escolhidas foram:

\begin{itemize}
    \item \textbf{Heurística não admissível (L1)} - Manhattan x 2
    \item \textbf{Heurística admissível (L2)} - Manhattan
    \item \textbf{Melhor Heurística admissível (L3)} - Manhattan + Conflitos Lineares
\end{itemize}

A heurística de Manhattan calcula a distância em que cada peça está posicionada e da sua posição destino (a posição correta) sem considerar as diagonais, portanto considerando deslocamentos apenas na vertical e horizontal. Essa heurística utilizada é considerada admissível pois não superestima o custo real, porém utilizá-la com o valor dobrado a torna não admissível pois superestima o custo real.

A utilização de Manhattan + Conflitos Lineares foi escolhida como a melhor heurística possível pois considera a distância de Manhattan, explicada anteriormente, e também leva em consideração a quantidade de peças na linha ou coluna correta mas em posição errada.

\section{Q4}

\paragraph{Proposta ---} enunciado na \textit{Questão 4}:

\begin{quote}
Breve análise do desempenho da implementação com uma tabela comparativa (usando as informações da saída - itens a) a d)) das 4 variações implementadas (no mínimo: um caso difícil, um médio e um fácil para as abordagens 3 e 4 e um médio e um fácil para as abordagens 1 e 2):
\end{quote}

\subsubsection{Solução --- \textbf{Q4}}

\paragraph{}
dkadkajaksd

\section{Q5}

\paragraph{Proposta ---} enunciado na \textit{Questão 5}:

\begin{quote}
Caso algum dos objetivos não tenha sido alcançado explique o que você faria VS o que foi feito e exatamente qual o(s)
  problema(s) encontrado(s), bem como  limitações da implementação:
\end{quote}

\subsubsection{Solução --- \textbf{Q5}}

\paragraph{}
Os objetivos foram alcançados, porém, foi necessária a otimização de como as estruturas estavam sendo usadas para armazenar as informações de execução das buscas. Como exemplo, tem-se a estruturação de como armazenar o caminho dos nós percorridos e abertos para casos onde há milhares de nós percorridos, causando memory leak caso a estrutura e método de armazenamento não sejam adequados.

\newpage
\section{Bibliografia}
\bibliographystyle{apalike}
\bibliography{references}

\end{document}

